\documentclass{sigcomm-alternate}

\begin{document}

\title{Net Neutrality - Does it Solve a Problem and What are the Advantages?}
	
\numberofauthors{3}

\author{
	% 1st. author
	\alignauthor
	Raz Friman\\
	\affaddr{Southern Methodist Univeristy}\\
	\affaddr{5600 SMU Blvd. APT 1516}\\
	\affaddr{Dallas, Texas, 75206}\\
	\email{rfriman@smu.edu}
	% 2nd. author
	\alignauthor
	Jarret Shook\\
	\affaddr{Southern Methodist Univeristy}\\
	\affaddr{5555 Amesbury Dr. APT 1303}\\
	\affaddr{Dallas, Texas, 75206}\\
	\email{jshook@smu.edu}
	% 3rd. author
	\alignauthor Elena Villamil\\
	\affaddr{Southern Methodist Univeristy}\\
	\affaddr{5555 Amesbury Dr. APT 1303}\\
	\affaddr{Dallas, Texas, 75206}\\
	\email{mvillamilrod@smu.edu}
	\and  % use '\and' if you need 'another row' of author names	
	% 4th. author
	\alignauthor Jeffrey Artigues\\
	\affaddr{Southern Mehthodist University}\\
	\affaddr{9520 Amberton Pkwy.}\\
	\affaddr{Dallas, Texas, 75243}\\
	\email{jartigues@smu.edu}
}

\maketitle


\section{Research Problem}
We plan to research the problem(s) related to net neutrality and the results of regulating the Internet as a utility. We will provide a clear definition to what net neutrality is, and approach the effects of having a net neutral environment. 

We will present two different sets of consequences: the effects of a net neutral network and the effects of it being regulated as a utility. First, we will focus on the major idea that in a net neutral world, there is no restriction on bandwidth usage. If this is the case, Internet providers will not be able to throttle back speeds based on how much network traffic is used. Therefore, we will analyze the recent and exponential increase of bandwidth usage due to video streaming, and any potential related byproducts of net neutrality. In addition, we will also research and provide examples of existing networks that are treated as utilities, with specific emphasis on the electrical and telephone networks. Using this information, we will analyze the effects that regulations have had on technological advancements in their respective fields. 

The goal of this research will be to draw conclusions on whether net neutrality and regulations would solve more problems than it introduces or vice versa.

     		
\section{Research Methodology}
We have several goals we wish to achieve in our research. First of all, we must define "Net Neutrality". Afterwards, we will explain some of the consequences of having a net neutral Internet. We will focus on the large amount of Internet bandwidth used by people today, its exponential growth, and the inability of Internet Service Providers (ISPs) to throttle back these connections under a net neutrality law. We will research the current usage of Internet bandwidth in the United States, focusing on the usage percentages of streaming video, web, email, etc. 

With permission, we would like to look into Southern Methodist University's network usage as a source of data for analysis. We will collect meta data about the traffic passing through the network. By analyzing this content, we can determine the type of usage of the data. This would tell us whether or not the data was used for burst traffic or as streaming traffic. Afterwards, we can use this sample set of data to extrapolate our findings to a broader level. 

We would also like to interview an employee at an ISP in the hopes of acquiring data regarding their service and bandwidth usage. We would like to use this research and data to conclude whether or not the current and future usage of the Internet would present an obstacle to the application of net neutrality. The following are some of the possible questions we will ask:

\begin{itemize}
	\item What percentage of your bandwidth is used by streaming traffic, web traffic, etc.
	\item During peak usage hours, what sort of interruptions in traffic do you experience, and how do you handle the times at which bandwidth usage exceeds your capacity
	\item How often do you experience peak network usage
	\item What is the bandwidth capacity of your Internet infrastructure
	\item If you have experienced any Denial-of-Service (DOS) attacks, how does it affect your service, and at what scale
\end{itemize}

Lastly, we will research current networks treated as utilities, such as telephone and electricity networks. We will research the changes in restrictiveness of the regulations imposed on these networks over time and attempt to correlate them to changes in the rate of innovation.


\section{Previous Research}
Although there is a multitude of information on the topic of the Internet and net neutrality, much of this information is opinionated or politically motivated. Once you start looking for unbiased, and technical research, the majority of information available drops. So far, in our preliminary research, we have found information on the following topics: 

\begin{itemize}
	\item "" talks about current percentages of bandwidth use. Import information form this the fact that nnnn of the current bandwidth usage is entertainment, adn this percentage is just expected to increase. According ot this articles on of the services that keep increasing its bandwidth usage is Netflix
	\item "FCC Chairman Tom Wheeler: This Is How We Will Ensure Net Neutrality" has a fairly good description of the new changes that the FCC is proposing regarding Internet and net neutrality.
	\item 
\end{itemize}


telecommunications act of 1934/1996 - opened up communication

\cite{FCCTomWheeler}
\cite{1224454}
\cite{1631969}
\cite{5188801}
\cite{5277804}


\section{Research Plan - MILESTONES AND GOALS}


A considerable amount of the time we have to complete this project will be spent researching. Having a solid research base before we start the process of compiling the information and writing the report is fundamental. Thus, with this in mind, we are planning on spending the first month of the three months we have to complete the project researching all the different topics mentioned above. Below we will detail the progress we expect to make during the first month:

\begin{itemize}	
	\item Finish researching bandwidth usage and net neutrality

	\item Start researching regulations on the electricity and phone networks

	\item We will work on getting approved to analyze SMU's network traffic. If given permission we will be collecting data until the end of the February
		
	\item Interview an employee at a major ISP (E.G. AT\&T or Time Warner Cable) and inquire about their bandwidth usage data
		
	\item We will also start working on the draft due on March 5th. Our goal is to have not only the required parts for that deliverable, but at least ninety percent of the parts required for this project. The more we have done at this point, the better the feedback we will receive% the more and better feedback we will get 
		
\end{itemize}



We will spend the rest of March working on polishing the paper and the research. We will focus heavily on any feedback we get from our initial draft. During this time we will also continue analyzing our research in order to draw our final conclusions.
\begin{itemize}
	\item Compile the data collected from the SMU network
	\item Analyze and extrapolate the data and trends found in the sample data from the SMU network
	\item Finalize interim draft
\end{itemize}


Finally, in April, we will begin to start working on our presentation. We intend to fully prepare the presentation at least in week in advanced in order to get feedback before we present to the class. 
\begin{itemize}
		\item Finalize the research paper
		\item Prepare the final presentation
\end{itemize}

		
\section{Resources Needed}
[TO PROOF READ]
In addition to the current information we have acquired, there are two main resources that would be very beneficial to our research. First, performing a network data analysis on an existing network, such as SMU's South Campus, could provide a very useful sample set of data about bandwidth usage. This data would allow us to perform statistical analysis and extrapolate future bandwidth usage that will be useful to draw conclusions about network usage on a broader scale. However, in order to have access to this data, we would need to get approved by the University. Secondly, being able to either interview or inquire information from an employee working at an ISP would allow us to gather much needed internal information about the ISP. Such as: how much capacity an ISP can support, the percentage of this bandwidth that is currently used on their network, and the percentage that remains free and available at any given time. Any sort of insider insight would help us form more informed conclusions about the internal capacity of the ISP's and how much bandwidth usage they can truly handle.


\bibliographystyle{unsrt}
\bibliography{bib}

\end{document}
