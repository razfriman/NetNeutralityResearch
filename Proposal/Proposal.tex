\documentclass{sigcomm-alternate}

\begin{document}

\title{Net Neutrality - Does it Solve a Problem and What are the Advantages?}
	
\numberofauthors{4}

\author{
	% 1st. author
	\alignauthor
	Raz Friman\\
	\affaddr{Southern Methodist Univeristy}\\
	\affaddr{5600 SMU Blvd. APT 1516}\\
	\affaddr{Dallas, Texas, 75206}\\
	\email{rfriman@smu.edu}
	% 2nd. author
	\alignauthor
	Jarret Shook\\
	\affaddr{Southern Methodist Univeristy}\\
	\affaddr{5555 Amesbury Dr. APT 1303}\\
	\affaddr{Dallas, Texas, 75206}\\
	\email{jshook@smu.edu}
	% 3rd. author
	\alignauthor Elena Villamil\\
	\affaddr{Southern Methodist Univeristy}\\
	\affaddr{5555 Amesbury Dr. APT 1303}\\
	\affaddr{Dallas, Texas, 75206}\\
	\email{mvillamilrod@smu.edu}
	\and  % use '\and' if you need 'another row' of author names	
	% 4th. author
	\alignauthor Jeffrey Artigues\\
	\affaddr{Brookhaven Laboratories}\\
	\affaddr{Brookhaven National Lab}\\
	\affaddr{Dallas, Texas, 75206}\\
	\email{jartigues@smu.edu}
}

\maketitle


\section{Research Problem}
We plan to research the problem(s) related to net neutrality and the results of regulating the Internet as a utility. We will provide a clear definition to what net neutrality is, and approach the effects of having a net neutral environment. 

We will present two different sets of consequences: the effects of a net neutral network and the effects of it being regulated as a utility. First, we will focus on the major idea that in a net neutral world, there is no restriction on bandwidth usage. If this is the case, Internet providers will not be able to throttle back speeds based on how much network traffic is used. Therefore, we will analyze the recent and exponential increase of bandwidth usage due to video streaming, and any potential related byproducts of net neutrality. In addition, we will also research and provide examples of existing networks that are treated as utilities, with specific emphasis on the electrical and telephone networks. Using this information, we will analyze the effects that regulations have had on technological advancements in their respective fields. 


The goal of this research will be to draw conclusions on whether current and future bandwidth usage are a problem and whether net neturality/regulation would help. regulating Internet as a utility would help solve current problems and whether government regulations net neutrality is a rational change to address a real problem.

%\cite{FCCTomWheeler}

     		
\section{Research Methodology}
We have several goals we wish to achieve in our research. First of all, we must define "Net Neutrality". Afterwards, we will attempt to explain the consequences of having a net neutral environment. With that said, we will focus on one large potential effect. This being the large amount of Internet bandwidth used by people today, and the inability of Internet Service Providers to throttle back these connections under a net neutrality law. We will research the current use of Internet bandwidth in the United States to analyze this effect reliably. This includes the usage of the bandwidth, such as the percentages of streaming video, web, email, etc. If we are approved we want to look into Southern Methodist University's network usage as a source of data for analysis. We would also like to call an Internet Service Provider (ISP) to interview them and try to get some data regarding their service and bandwidth usage. We would like to use this research and data to conclude whether the current and future usage of Internet present a problem or not to net neutrality.  

Lastly, we want to research and determine if net neutrality is the solution to any existing or future problems. In the case that in the first part of our research we conclude that the current Internet usage is not and will not be a problem.

[TALK ABOUT THE REGULATING AS A UTILITY AND WHAT WE PLAN TO DO FOR THAT AS WELL.]

\section{Previous Research}
Although there is a multitude of information on the topic of the Internet and net neutrality, much of this information is opinionated or politically motivated. Once you start looking for true, technical, raw research, the majority of information available drops. So far, we have found analysis and data on bandwidth usage, and some research papers on net neutrality. We will be using the bandwidth information to address the first major topic we are discussing, which is a potential problem that could affect net neutral environments. [THAT LAST PART MAY NOT BE ACCURATE, AND WE PROBABLY WANT TO EXPLAIN A LITTLE MORE OF WHAT EACH ONE OF THIS SOURCES WOULD CONTRIBUTE TO OUR RESEARCH].
TODO
\cite{Jain:2002:EAB:633025.633054}
\cite{Strauss:2003:MSA:948205.948211}



\section{Research Plan - MILESTONES AND GOALS}

\begin{itemize}
	\item A considerable amount of the time we have to complete this project will be expend researching. Having a solid research before we start the process of compiling the information and writing the report is fundamental. Thus, with this in mind we are planning on spending the first month of the three months we have to complete the project researching all the different topics mention above. 
	
	\item During this first month we will also work on getting approved to analyze SMU's network traffic. If given permission we will be collecting data until the end of the first month. Thus, by the end of February we aim to have most of the research done, the interview to some ISP and the data collected from SMU's network. 
	
	\item On February 20th we will start working on the draft due on March 5th. Our goal is to have not just the required parts for that deliverable, but at least ninety percent of the parts required for this project. The more we have done at this point, the more and better feedback we will get. We will spend the rest of March working on polishing the paper; we will specially focus on whatever feedback we get from the draft. Finally, around April 16th we will also start working on the presentation. We intent to have it done a few days before it is done so we can get some feedback on it before we present.  
\end{itemize}






\section{Resources Needed}
In addition to the current information we have acquired, there are two main resources that would be very beneficial to our research. First, performing a network data analysis on an existing network. We believe SMU's South Campus would provide a very useful sample set of data. This data would allow us to perform statistical analysis in order to come up with conclusions about network usage. However, in order to have access to this data, we would need to get approved by the University. Secondly, being able to either interview or inquire information from an employee working at an ISP would allow us to gather much needed information. Such as, how much capacity an ISP can support, the percentage of this bandwidth that is currently used on their network, and the percentage that remains free and available at any given time. Any sort of insider insight would help us form more informed conclusions about the internal capacity of the ISP's and how much bandwidth they can truly support.



\bibliographystyle{unsrt}
\bibliography{bib}

\end{document}